We presented \nameNS, a Byzantine fault tolerant consensus protocol for ordering transactions that ensures the resulting order was \emph{fairly} determined. We believe that our work is highly suitable for ledger implementations in which transaction fees are not the sole motivation for processing transactions, such as decentralized ledgers whose nodes are operated by companies, each wishing to service its own users. Indeed, such use-case requires the protocol to achieve a fair ordering of transactions, where all participants experience an equal level of service (transaction confirmation time, throughput, etc.). \name is providing this property while keeping the decentralized control of the ledger in the hands of the participating nodes; the capability to scale the transaction throughput; and a Byzantine fault tolerant engine. \name ensures the end-users enhanced protection from censorship and discrimination relative to other typical solutions, while requiring them to utilize a negligible amount of resources. It can therefore be seen as a fair protocol towards end-users as well.

%We presented \nameNS, a Byzantine fault tolerant consensus protocol for ordering transactions that realizes a \emph{fair} order of the transactions it processes. We believe that our work is highly suitable for a cryptocurrency implementation that assumes nodes are motivated to include transactions in blocks for reasons other than transaction fees, e.g., nodes run applications and wish to service their own users. Indeed, such a use-case would require the protocol to achieve a fair ordering of transactions, where all applications enjoy an equal level of service (transaction confirmation time, throughput, etc.). Helix addresses this property together with maintaining, decentralized control over the ledger by the participating nodes; scalability capabilities in the throughput of transactions; and a Byzantine fault tolerance engine. Finally, while users are required to use negligible resources, they enjoy enhanced protection from censorship and discrimination relative to other typical solutions. Thus, Helix can be seen as a fair protocol towards users.

%We presented \nameNS, as a fair and scalable algorithm for transactions ordering. \name makes use of bounded-size committees to reach consensus on the order while allowing scalability. \name relies on an efficient mechanism for block selection and validation to achieve fairness among network nodes. This guarantees for each of the network nodes representation of its transactions in a rate similar to their issuance rate. We also describe incentives that can further enhance fairness and can encourage nodes to follow more strictly different aspects of their expected behavior. Our experiments show that \name achieves fairness with a limited ability for various nodes to avoid it.  