Processing a transaction in Gruffalo consumes resources - bandwidth, storage and compute - and thus costs money. We assume that it is valuable for a node to fund its users transactions' processing costs. However, a node needs to pay for sufficient resources to process \textbf{all} nodes' transactions. In the simplest scenario, where all nodes produce the same amount of transactions, this is settled trivially as costs balance. In a more realistic scenario, where nodes produce different amount of transactions, a fee mechanism that balances costs among the nodes is required. Fees would basically be paid from nodes with a high transaction rate to those with a low transaction rate. 

Fees, in addition to balancing costs, serve as means to incentivize nodes to follow the protocol's instructions and allow it to reach optimal performance. This is achieved by a reputation measure that is attributed to each node and is updated frequently. The reputation strives to serve as a reliable measure to a node's quality of service. The fees paid to a node would be corrected according to its reputation where, nodes with low reputation will be paid less than their relative share, resulting in losses. Thus, it is crucial that Gruffalo's fee mechanism will measure accurately the usage of each node and that the reputation will measure correctly nodes' quality of service. %(or: resources and behavior)

%Gruffalo's fee mechanism is based on a trusted third party\footnote{This entity can be represented by a smart contract\red{, in a different platform or over Gruffalo itself}.} that collects fees from nodes and pays them back later taking into account their issuance rate and reputation. In practice the third party would transfer funds from nodes with high usage to those with low usage taking into account nodes' reputation. Nodes with low reputation will be paid back less than their relative share, resulting in losses. Thus, it is crucial that Gruffalo's fee mechanism will measure accurately the usage of each node and that the reputation will measure correctly nodes' resources and behavior.

Measuring a node's usage is quite an easy task as all $etx$s' owner nodes are revealed after the decryption process. On the contrary, making sure the reputation is consistent with Gruffalo's requirements and instructions is more complicated. We shall suggest a few practical criteria that the reputation may check, but leave a more elaborate discussion as to its exact specification to a later time. We further note that with the evolution of the system, the reputation measure may be upgraded in order to mitigate new found vulnerabilities in the protocol or to encourage new desired behaviors.

To conclude this discussion we emphasize that nodes are incentivized to increase their reputation and also wish to include their own $etx$s in Eblocks as fast as possible. In addition, at this point we do not elaborate as to the implementation of the fee structure in Gruffalo, but emphasize that a fee-per-transaction will be enforced, where each transaction will be paid for by its owner node. 

We shall now illustrate a naive reputation measure. First, the reputation measure will be updated frequently (e.g. every 1000 blocks or every month), locally and deterministically by each node, according to the blockchain. A few concrete examples:
\begin{itemize}
\item In order to make sure that nodes do not delay the protocol, every node will be evaluated according to the average view number of committed Eblocks when it was a committee member. The higher the view, the lower the evaluation should be.
\item In order to make sure that nodes maintain the recent blockchain history, if an Eblock consists of a previously included $etx$, it will be punished by reducing the composer's reputation.
\item In order to make sure that nodes select $etx$s randomly (see Sec.~\ref{section_highload} for more details), adjacent Eblocks should represent similar distributions (in terms of the owner nodes of the $etx$s they include). If a specific node was found to construct Eblocks that significantly deviate from the Eblocks in its surrounding, it will be punished by reducing its reputation.
\item In order to encourage nodes to upgrade the software and stay up to date, a node that constructed an Eblock with an old version will be punished by reducing its reputation.
\end{itemize}

We emphasize that the reputation update rules that do not depend on the decrypted-blocks may be hardened and become validation checks. On the contrary, some of the validation checks in~\ref{} may be relaxed and become a part of the reputation mechanism. These two mechanisms share a common goal - to have the blockchain represent nodes' needs and desires. An interesting possibility to consider is to tie reputation with a node's stake in the system.
